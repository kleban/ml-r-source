% Options for packages loaded elsewhere
\PassOptionsToPackage{unicode}{hyperref}
\PassOptionsToPackage{hyphens}{url}
\PassOptionsToPackage{dvipsnames,svgnames,x11names}{xcolor}
%
\documentclass[
  letterpaper,
  DIV=11,
  numbers=noendperiod]{scrreprt}

\usepackage{amsmath,amssymb}
\usepackage{lmodern}
\usepackage{iftex}
\ifPDFTeX
  \usepackage[T1]{fontenc}
  \usepackage[utf8]{inputenc}
  \usepackage{textcomp} % provide euro and other symbols
\else % if luatex or xetex
  \usepackage{unicode-math}
  \defaultfontfeatures{Scale=MatchLowercase}
  \defaultfontfeatures[\rmfamily]{Ligatures=TeX,Scale=1}
\fi
% Use upquote if available, for straight quotes in verbatim environments
\IfFileExists{upquote.sty}{\usepackage{upquote}}{}
\IfFileExists{microtype.sty}{% use microtype if available
  \usepackage[]{microtype}
  \UseMicrotypeSet[protrusion]{basicmath} % disable protrusion for tt fonts
}{}
\makeatletter
\@ifundefined{KOMAClassName}{% if non-KOMA class
  \IfFileExists{parskip.sty}{%
    \usepackage{parskip}
  }{% else
    \setlength{\parindent}{0pt}
    \setlength{\parskip}{6pt plus 2pt minus 1pt}}
}{% if KOMA class
  \KOMAoptions{parskip=half}}
\makeatother
\usepackage{xcolor}
\setlength{\emergencystretch}{3em} % prevent overfull lines
\setcounter{secnumdepth}{5}
% Make \paragraph and \subparagraph free-standing
\ifx\paragraph\undefined\else
  \let\oldparagraph\paragraph
  \renewcommand{\paragraph}[1]{\oldparagraph{#1}\mbox{}}
\fi
\ifx\subparagraph\undefined\else
  \let\oldsubparagraph\subparagraph
  \renewcommand{\subparagraph}[1]{\oldsubparagraph{#1}\mbox{}}
\fi

\usepackage{color}
\usepackage{fancyvrb}
\newcommand{\VerbBar}{|}
\newcommand{\VERB}{\Verb[commandchars=\\\{\}]}
\DefineVerbatimEnvironment{Highlighting}{Verbatim}{commandchars=\\\{\}}
% Add ',fontsize=\small' for more characters per line
\usepackage{framed}
\definecolor{shadecolor}{RGB}{241,243,245}
\newenvironment{Shaded}{\begin{snugshade}}{\end{snugshade}}
\newcommand{\AlertTok}[1]{\textcolor[rgb]{0.68,0.00,0.00}{#1}}
\newcommand{\AnnotationTok}[1]{\textcolor[rgb]{0.37,0.37,0.37}{#1}}
\newcommand{\AttributeTok}[1]{\textcolor[rgb]{0.40,0.45,0.13}{#1}}
\newcommand{\BaseNTok}[1]{\textcolor[rgb]{0.68,0.00,0.00}{#1}}
\newcommand{\BuiltInTok}[1]{\textcolor[rgb]{0.00,0.23,0.31}{#1}}
\newcommand{\CharTok}[1]{\textcolor[rgb]{0.13,0.47,0.30}{#1}}
\newcommand{\CommentTok}[1]{\textcolor[rgb]{0.37,0.37,0.37}{#1}}
\newcommand{\CommentVarTok}[1]{\textcolor[rgb]{0.37,0.37,0.37}{\textit{#1}}}
\newcommand{\ConstantTok}[1]{\textcolor[rgb]{0.56,0.35,0.01}{#1}}
\newcommand{\ControlFlowTok}[1]{\textcolor[rgb]{0.00,0.23,0.31}{#1}}
\newcommand{\DataTypeTok}[1]{\textcolor[rgb]{0.68,0.00,0.00}{#1}}
\newcommand{\DecValTok}[1]{\textcolor[rgb]{0.68,0.00,0.00}{#1}}
\newcommand{\DocumentationTok}[1]{\textcolor[rgb]{0.37,0.37,0.37}{\textit{#1}}}
\newcommand{\ErrorTok}[1]{\textcolor[rgb]{0.68,0.00,0.00}{#1}}
\newcommand{\ExtensionTok}[1]{\textcolor[rgb]{0.00,0.23,0.31}{#1}}
\newcommand{\FloatTok}[1]{\textcolor[rgb]{0.68,0.00,0.00}{#1}}
\newcommand{\FunctionTok}[1]{\textcolor[rgb]{0.28,0.35,0.67}{#1}}
\newcommand{\ImportTok}[1]{\textcolor[rgb]{0.00,0.46,0.62}{#1}}
\newcommand{\InformationTok}[1]{\textcolor[rgb]{0.37,0.37,0.37}{#1}}
\newcommand{\KeywordTok}[1]{\textcolor[rgb]{0.00,0.23,0.31}{#1}}
\newcommand{\NormalTok}[1]{\textcolor[rgb]{0.00,0.23,0.31}{#1}}
\newcommand{\OperatorTok}[1]{\textcolor[rgb]{0.37,0.37,0.37}{#1}}
\newcommand{\OtherTok}[1]{\textcolor[rgb]{0.00,0.23,0.31}{#1}}
\newcommand{\PreprocessorTok}[1]{\textcolor[rgb]{0.68,0.00,0.00}{#1}}
\newcommand{\RegionMarkerTok}[1]{\textcolor[rgb]{0.00,0.23,0.31}{#1}}
\newcommand{\SpecialCharTok}[1]{\textcolor[rgb]{0.37,0.37,0.37}{#1}}
\newcommand{\SpecialStringTok}[1]{\textcolor[rgb]{0.13,0.47,0.30}{#1}}
\newcommand{\StringTok}[1]{\textcolor[rgb]{0.13,0.47,0.30}{#1}}
\newcommand{\VariableTok}[1]{\textcolor[rgb]{0.07,0.07,0.07}{#1}}
\newcommand{\VerbatimStringTok}[1]{\textcolor[rgb]{0.13,0.47,0.30}{#1}}
\newcommand{\WarningTok}[1]{\textcolor[rgb]{0.37,0.37,0.37}{\textit{#1}}}

\providecommand{\tightlist}{%
  \setlength{\itemsep}{0pt}\setlength{\parskip}{0pt}}\usepackage{longtable,booktabs,array}
\usepackage{calc} % for calculating minipage widths
% Correct order of tables after \paragraph or \subparagraph
\usepackage{etoolbox}
\makeatletter
\patchcmd\longtable{\par}{\if@noskipsec\mbox{}\fi\par}{}{}
\makeatother
% Allow footnotes in longtable head/foot
\IfFileExists{footnotehyper.sty}{\usepackage{footnotehyper}}{\usepackage{footnote}}
\makesavenoteenv{longtable}
\usepackage{graphicx}
\makeatletter
\def\maxwidth{\ifdim\Gin@nat@width>\linewidth\linewidth\else\Gin@nat@width\fi}
\def\maxheight{\ifdim\Gin@nat@height>\textheight\textheight\else\Gin@nat@height\fi}
\makeatother
% Scale images if necessary, so that they will not overflow the page
% margins by default, and it is still possible to overwrite the defaults
% using explicit options in \includegraphics[width, height, ...]{}
\setkeys{Gin}{width=\maxwidth,height=\maxheight,keepaspectratio}
% Set default figure placement to htbp
\makeatletter
\def\fps@figure{htbp}
\makeatother
\newlength{\cslhangindent}
\setlength{\cslhangindent}{1.5em}
\newlength{\csllabelwidth}
\setlength{\csllabelwidth}{3em}
\newlength{\cslentryspacingunit} % times entry-spacing
\setlength{\cslentryspacingunit}{\parskip}
\newenvironment{CSLReferences}[2] % #1 hanging-ident, #2 entry spacing
 {% don't indent paragraphs
  \setlength{\parindent}{0pt}
  % turn on hanging indent if param 1 is 1
  \ifodd #1
  \let\oldpar\par
  \def\par{\hangindent=\cslhangindent\oldpar}
  \fi
  % set entry spacing
  \setlength{\parskip}{#2\cslentryspacingunit}
 }%
 {}
\usepackage{calc}
\newcommand{\CSLBlock}[1]{#1\hfill\break}
\newcommand{\CSLLeftMargin}[1]{\parbox[t]{\csllabelwidth}{#1}}
\newcommand{\CSLRightInline}[1]{\parbox[t]{\linewidth - \csllabelwidth}{#1}\break}
\newcommand{\CSLIndent}[1]{\hspace{\cslhangindent}#1}

\KOMAoption{captions}{tableheading}
\makeatletter
\@ifpackageloaded{tcolorbox}{}{\usepackage[many]{tcolorbox}}
\@ifpackageloaded{fontawesome5}{}{\usepackage{fontawesome5}}
\definecolor{quarto-callout-color}{HTML}{909090}
\definecolor{quarto-callout-note-color}{HTML}{0758E5}
\definecolor{quarto-callout-important-color}{HTML}{CC1914}
\definecolor{quarto-callout-warning-color}{HTML}{EB9113}
\definecolor{quarto-callout-tip-color}{HTML}{00A047}
\definecolor{quarto-callout-caution-color}{HTML}{FC5300}
\definecolor{quarto-callout-color-frame}{HTML}{acacac}
\definecolor{quarto-callout-note-color-frame}{HTML}{4582ec}
\definecolor{quarto-callout-important-color-frame}{HTML}{d9534f}
\definecolor{quarto-callout-warning-color-frame}{HTML}{f0ad4e}
\definecolor{quarto-callout-tip-color-frame}{HTML}{02b875}
\definecolor{quarto-callout-caution-color-frame}{HTML}{fd7e14}
\makeatother
\makeatletter
\makeatother
\makeatletter
\@ifpackageloaded{bookmark}{}{\usepackage{bookmark}}
\makeatother
\makeatletter
\@ifpackageloaded{caption}{}{\usepackage{caption}}
\AtBeginDocument{%
\ifdefined\contentsname
  \renewcommand*\contentsname{Зміст}
\else
  \newcommand\contentsname{Зміст}
\fi
\ifdefined\listfigurename
  \renewcommand*\listfigurename{List of Figures}
\else
  \newcommand\listfigurename{List of Figures}
\fi
\ifdefined\listtablename
  \renewcommand*\listtablename{List of Tables}
\else
  \newcommand\listtablename{List of Tables}
\fi
\ifdefined\figurename
  \renewcommand*\figurename{Рис.}
\else
  \newcommand\figurename{Рис.}
\fi
\ifdefined\tablename
  \renewcommand*\tablename{Таблиця}
\else
  \newcommand\tablename{Таблиця}
\fi
}
\@ifpackageloaded{float}{}{\usepackage{float}}
\floatstyle{ruled}
\@ifundefined{c@chapter}{\newfloat{codelisting}{h}{lop}}{\newfloat{codelisting}{h}{lop}[chapter]}
\floatname{codelisting}{Лістинг}
\newcommand*\listoflistings{\listof{codelisting}{List of Listings}}
\makeatother
\makeatletter
\@ifpackageloaded{caption}{}{\usepackage{caption}}
\@ifpackageloaded{subcaption}{}{\usepackage{subcaption}}
\makeatother
\makeatletter
\@ifpackageloaded{tcolorbox}{}{\usepackage[many]{tcolorbox}}
\makeatother
\makeatletter
\@ifundefined{shadecolor}{\definecolor{shadecolor}{rgb}{.97, .97, .97}}
\makeatother
\makeatletter
\makeatother
\ifLuaTeX
  \usepackage{selnolig}  % disable illegal ligatures
\fi
\IfFileExists{bookmark.sty}{\usepackage{bookmark}}{\usepackage{hyperref}}
\IfFileExists{xurl.sty}{\usepackage{xurl}}{} % add URL line breaks if available
\urlstyle{same} % disable monospaced font for URLs
\hypersetup{
  pdftitle={Основи роботи з даними в R},
  pdfauthor={Юрій Клебан},
  colorlinks=true,
  linkcolor={blue},
  filecolor={Maroon},
  citecolor={Blue},
  urlcolor={Blue},
  pdfcreator={LaTeX via pandoc}}

\title{Основи роботи з даними в R}
\usepackage{etoolbox}
\makeatletter
\providecommand{\subtitle}[1]{% add subtitle to \maketitle
  \apptocmd{\@title}{\par {\large #1 \par}}{}{}
}
\makeatother
\subtitle{Навчальний посібник з курсу \textbf{Аналіз даних в \texttt{R}}
для студентів спеціальності економічна кібернетика, фінанси, 2 курс}
\author{Юрій Клебан}
\date{Invalid Date}

\begin{document}
\maketitle
\ifdefined\Shaded\renewenvironment{Shaded}{\begin{tcolorbox}[enhanced, borderline west={3pt}{0pt}{shadecolor}, breakable, frame hidden, interior hidden, sharp corners, boxrule=0pt]}{\end{tcolorbox}}\fi

\renewcommand*\contentsname{Зміст}
{
\hypersetup{linkcolor=}
\setcounter{tocdepth}{2}
\tableofcontents
}
\bookmarksetup{startatroot}

\hypertarget{ux43fux440ux43e-ux43fux43eux441ux456ux431ux43dux438ux43a}{%
\chapter*{Про
посібник}\label{ux43fux440ux43e-ux43fux43eux441ux456ux431ux43dux438ux43a}}
\addcontentsline{toc}{chapter}{Про посібник}

Юрій Клебан\\
2023-03-19

\hfill\break

\begin{figure}

{\centering 

\href{https://doi.org/10.5281/zenodo.7251419}{\includegraphics{https://zenodo.org/badge/DOI/10.5281/zenodo.7251419.svg}}

}

\end{figure}

Замінити інформацію про курс Аналіз даних

Матеріали навчального посібника підготовлені для читання курсу
\textbf{\emph{``Вступ до прикладного програмування в R''}}
\texttt{{[}05.250{]}} студентам 1-го року навчання спеціальності
економічна кібернетика \href{https://oa.edu.ua}{Національного
університету ``Острозька академія''}.

\hypertarget{ux43eux43fux438ux441-ux43dux430ux432ux447ux430ux43bux44cux43dux43eux457-ux434ux438ux441ux446ux438ux43fux43bux456ux43dux438}{%
\section*{Опис навчальної
дисципліни}\label{ux43eux43fux438ux441-ux43dux430ux432ux447ux430ux43bux44cux43dux43eux457-ux434ux438ux441ux446ux438ux43fux43bux456ux43dux438}}
\addcontentsline{toc}{section}{Опис навчальної дисципліни}

Навчальна дисципліна спрямована на вивчення основ практичного
застосування популярної мови R для проведення статистичних досліджень в
економіці.

У процесі вивчення курсу розглядаються теми, що стосуються теоретичних
основ та практичної реалізації алгоритмів, завантаження, підготовки та
обробки економічних даних.

Місце навчальної дисципліни у підготовці здобувачів: програмні
результати дисципліни використовуються під час вивчення таких навчальних
дисциплін: ``Алгоритми та структури даних'', ``Аналіз даних в R'',
``Прикладне математичне моделювання в R'', ``Підготовка аналітичних
звітів''. Закріплення на практиці здобутих програмних результатів
відбувається під час проходження навчальної практики з курсу
``Економіко-математичне моделювання''.

\hypertarget{ux43cux435ux442ux430-ux434ux438ux441ux446ux438ux43fux43bux456ux43dux438}{%
\section*{Мета
дисципліни}\label{ux43cux435ux442ux430-ux434ux438ux441ux446ux438ux43fux43bux456ux43dux438}}
\addcontentsline{toc}{section}{Мета дисципліни}

Мета навчальної дисципліни -- формування у студентів теоретичних знань
та практичних навичок використання мови програмування R для роботи з
даними та базовими структурами мови (типи даних, розгалуження, цикли,
функції).

\begin{center}\rule{0.5\linewidth}{0.5pt}\end{center}

\hypertarget{ux43fux456ux434ux442ux440ux438ux43cux43aux430-ux43fux440ux43eux454ux43aux442ux443}{%
\section*{Підтримка
проєкту}\label{ux43fux456ux434ux442ux440ux438ux43cux43aux430-ux43fux440ux43eux454ux43aux442ux443}}
\addcontentsline{toc}{section}{Підтримка проєкту}

Матеріали навчального посібника створено у межах проєкту ``Підготовка,
обробка та ефективне використання даних для наукових досліджень (на
основі R)'', що підтримується Європейським союзою за програмою
\href{https://houseofeurope.org.ua/}{House of Europe}.

~~

\hypertarget{ux434ux43eux442ux440ux438ux43cux430ux43dux43dux44f-ux43fux440ux438ux43dux446ux438ux43fux456ux432-ux434ux43eux431ux440ux43eux447ux435ux441ux43dux43eux441ux442ux456}{%
\section*{Дотримання принципів
доброчесності}\label{ux434ux43eux442ux440ux438ux43cux430ux43dux43dux44f-ux43fux440ux438ux43dux446ux438ux43fux456ux432-ux434ux43eux431ux440ux43eux447ux435ux441ux43dux43eux441ux442ux456}}
\addcontentsline{toc}{section}{Дотримання принципів доброчесності}

Викладач та слухач цього курсу, як очікується, повинні дотримуватися
Кодексу академічної доброчесності університету:

\begin{itemize}
\item
  будь-яка робота, подана здобувачем протягом курсу, має бути його
  власною роботою здобувача; не вдаватися до кроків, що можуть нечесно
  покращити Ваші результати чи погіршити/покращити результати інших
  здобувачів;
\item
  якщо буде виявлено ознаки плагіату або іншої недобросовісної
  академічної поведінки, то студент буде позбавлений можливості отримати
  передбачені бали за завдання;
\item
  не публікувати у відкритому доступі відповіді на запитання, що
  використовуються в рамках курсу для оцінювання знань здобувачів;
\item
  під час фінальних видів контролю необхідно працювати самостійно; не
  дозволяється говорити або обговорювати, а також не можна копіювати
  документи, використовувати електронні засоби отримання інформації.
\end{itemize}

Порушення академічної доброчесності під час виконання контрольних
завдань призведе до втрати балів або вживання заходів, які передбачені
Кодексу академічної доброчесності НаУОА.

\begin{center}\rule{0.5\linewidth}{0.5pt}\end{center}

\begin{tcolorbox}[enhanced jigsaw, opacityback=0, breakable, arc=.35mm, left=2mm, toprule=.15mm, colframe=quarto-callout-note-color-frame, colback=white, leftrule=.75mm, rightrule=.15mm, bottomrule=.15mm]
\begin{minipage}[t]{5.5mm}
\textcolor{quarto-callout-note-color}{\faInfo}
\end{minipage}%
\begin{minipage}[t]{\textwidth - 5.5mm}

Матеріали курсу створені з використанням ряду технологій та середовищ
розробки:

\begin{itemize}
\item[$\boxtimes$]
  \href{https://www.r-project.org}{Мова \texttt{R}} - безкоштована мова
  програмування для виконання досліджень у сфері статистики, машинного
  навчання та візуалізацї результатів.
\item[$\boxtimes$]
  \href{https://quarto.org}{\texttt{Quarto} Book} - система для
  публікації наукових та технічних текстів з відкритим кодом
  (\texttt{R}/\texttt{Python}/\texttt{Julia}/\texttt{Observable}).
\item[$\boxtimes$]
  \href{https://github.com/jupyterlab/jupyterlab}{\texttt{JupyterLab}} -
  середовище розробки на основі
  \href{https://jupyter.org/}{\texttt{Jupyter\ Notebook}}.
  \texttt{JupyterLab} є розширеним веб-інтерфейсом для роботи з
  ноутбуками.
\item[$\boxtimes$]
  \href{https://git-scm.com/}{\texttt{Git}}/\href{https://github.com/}{\texttt{Github}}
  - система контролю версій та, відповідно, сервіс для організації
  зберігання коду, а також публікації статичних сторінок.
\item[$\boxtimes$]
  \href{https://www.rstudio.com/}{\texttt{RStudio\ Desktop}} -
  інтегроване середовище розробки (\texttt{IDE}) для мови \texttt{R} з
  відкритим кодом, що містить в собі редактор коду, консоль, планер,
  засоби візуалізації та можливості.
\item[$\boxtimes$]
  \href{https://code.visualstudio.com/}{\texttt{Visual\ Studio\ Code}} -
  інтегроване середовище розробки (\texttt{IDE}) з відкритим кодом
  практично для усіх відомих технологій та мов програмування.
\end{itemize}

\end{minipage}%
\end{tcolorbox}

\begin{center}\rule{0.5\linewidth}{0.5pt}\end{center}

Бібілографічний опис \texttt{bibtex}:

\begin{Shaded}
\begin{Highlighting}[]
\VariableTok{@book}\NormalTok{\{}\OtherTok{yk}\NormalTok{{-}}\OtherTok{r}\NormalTok{{-}}\OtherTok{intro}\NormalTok{,}
  \DataTypeTok{author}\NormalTok{       = \{Юрій Клебан\},}
  \DataTypeTok{title}\NormalTok{        = \{Вступ до програмування в R\},}
  \DataTypeTok{publisher}\NormalTok{    = \{Zenodo\},}
  \DataTypeTok{year}\NormalTok{         = 2022,}
  \DataTypeTok{doi}\NormalTok{          = \{10.5281/zenodo.7251419\},}
  \DataTypeTok{url}\NormalTok{          = \{https://doi.org/10.5281/zenodo.7251419\}}
\NormalTok{\}}
\end{Highlighting}
\end{Shaded}

\bookmarksetup{startatroot}

\hypertarget{ux432ux441ux442ux443ux43f}{%
\chapter*{Вступ}\label{ux432ux441ux442ux443ux43f}}
\addcontentsline{toc}{chapter}{Вступ}

Юрій Клебан\\
2023-03-19

\hfill\break

Замінити вступ! ----

Фахівці спеціальності економічна кібернетика, а також фінанси та кредит
у майбутньому працюватимуть з великими масивами даних, що накопичуються
у даний момент і збиралися у попередні дисятиліття. Підготовка, обробка
і трансформація даних у зручний формат прийняття рішень забирає все
більше часу, а звичні рашіне інструменти аналізу даних, як наприклад,
\texttt{Microsoft\ Excel} не мають достатньо вбудованих можливостей для
виконнання задач бізнесу.

На даний час існує велика кількість мов програмування, що інтегруються у
суспільні сфери діяльності людини та роботи технічних систем:
біоінформатика, а також економіка та бізнес.

Однією з мов програмування, що отримали широке поширення серед
економістів-науковців, аналітиків та практиків математичного моделювання
(\texttt{machine\ learning}) є мова програмування \texttt{R}(R Core Team
2020). Свою популярність ця мова програмування здобула завдяки простоті
у використанні, доступності (безкоштовні як базові компоненти для
написання коду, так і середовища розробки), розширюваності (кожен
розробник має можливість створювати власні пакети та публікувати їх у
відкритому доступі).

Основними задачами курсу ``Вступ до прикладного програмування в R'' є
ознайомлення студентів з базовми конструкціями мови програмування
\texttt{R}, вивчення способів роботи з найпоширенішими типами даних,
читання інформації з різноманітних джерел. Також студенти отримують
знання про можливості використання \texttt{R} для виконання задач
аналізу даних та візуалізації.

\begin{center}\rule{0.5\linewidth}{0.5pt}\end{center}

\bookmarksetup{startatroot}

\hypertarget{ux441ux43fux438ux441ux43eux43a-ux432ux438ux43aux43eux440ux438ux441ux442ux430ux43dux438ux445-ux434ux436ux435ux440ux435ux43b}{%
\chapter*{Список використаних
джерел}\label{ux441ux43fux438ux441ux43eux43a-ux432ux438ux43aux43eux440ux438ux441ux442ux430ux43dux438ux445-ux434ux436ux435ux440ux435ux43b}}
\addcontentsline{toc}{chapter}{Список використаних джерел}

\hypertarget{refs}{}
\begin{CSLReferences}{1}{0}
\leavevmode\vadjust pre{\hypertarget{ref-r-base}{}}%
R Core Team. 2020. \emph{R: A Language and Environment for Statistical
Computing}. Vienna, Austria: R Foundation for Statistical Computing.
\url{https://www.R-project.org/}.

\end{CSLReferences}



\end{document}
